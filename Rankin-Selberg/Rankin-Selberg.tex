\documentclass[12pt, letterpaper, twoside]
{article}

\usepackage{amsmath}
\usepackage{amsfonts}
\usepackage{amssymb}
\usepackage{mathrsfs}
\usepackage[shortlabels]{enumitem}
\usepackage{tikz-cd}

\title{Rankin-Selberg Method}
\author{Nick Pilotti}

% Categories
\newcommand{\ra}{\rightarrow}
\newcommand{\xra}{\xrightarrow} % labeled arrow
\newcommand{\Id}{{\text{Id}}} % the identity

% Sets
\newcommand{\sub}{{\; \subseteq \;}} % subset relation
\newcommand{\super}{{\; \supseteq \;}} % superset relation
\newcommand{\pow}{{\mathcal{P}}} % power set

% Combinatorics
\def\multiset#1#2{\ensuremath{\left(\kern-.3em\left(\genfrac{}{}{0pt}{}{#1}{#2}\right)\kern-.3em\right)}}

% Number systems
\newcommand{\N}{{\mathbb N}} % natural numbers
\newcommand{\Z}{{\mathbb Z}} % integers
\newcommand{\Q}{{\mathbb Q}} % rationals
\newcommand{\R}{{\mathbb R}} % reals
\newcommand{\C}{{\mathbb C}} % complex numbers
\newcommand{\F}{{\mathbb F}} % a finite field

% Topology
\newcommand{\Bd}{\text{Bd}} % boundary
\newcommand{\Int}{\text{Int}} % interior

% Algebraic Topology
\newcommand*\sq{\mathbin{\vcenter{\hbox{\rule{.3ex}{.3ex}}}}} % concatenation

% Geometry
\newcommand{\Proj}{\mathbb{P}} % projective space

% Metric Spaces
\newcommand{\D}{\mathbb{D}} % unit disk
\newcommand{\B}{\mathbb{B}} % unit ball

% Measure Theory
\newcommand{\meas}{\mathcal{M}} % measurable sets
\newcommand{\borel}{\mathcal{B}} % borel sets

% Analysis
\newcommand{\dd}[1]{\mathrm{d}#1} % differential
\newcommand{\del}{\partial} % partial derivative

% Complex Analysis
\newcommand{\res}{\text{res}} % residue

% General Algebra
\newcommand{\Ker}{{\text{Ker}}} % kernel
\newcommand{\Coker}{{\text{Coker}}} % cokernel
\newcommand{\coker}{{\text{coker}}} % cokernel
\newcommand{\Img}{{\text{Im}}} % image
\newcommand{\img}{{\text{im}}} % image
\newcommand{\Hom}{\text{Hom}} % the hom functor
\newcommand{\End}{{\text{End}}} % endomorphisms
\newcommand{\Aut}{{\text{Aut}}} % automorphisms

% Group Theory
\newcommand{\Orb}{\text{Orb}} % orbit
\newcommand{\Stab}{{\text{Stab}}} % stabilizer
\newcommand{\nor}{{\trianglelefteq} \; } % normal subgroup
\newcommand{\Syl}{{\text{Syl}}} % sylow subgroups

% Specific groups
\newcommand{\Sym}{{\text{Sym}}} % symmetric group
\newcommand{\GL}{{\text{GL}}} % general linear group
\newcommand{\PGL}{{\text{PGL}}} % projective general linear group
\newcommand{\SL}{{\text{SL}}} % special linear group
\newcommand{\PSL}{{\text{PSL}}} % projective special linear group

% Rings
\newcommand{\cha}{\text{char}} % characteristic
\newcommand{\aI}{{\mathfrak{a}}} % an ideal
\newcommand{\bI}{{\mathfrak{b}}} % an ideal
\newcommand{\cI}{{\mathfrak{c}}} % an ideal
\newcommand{\pI}{{\mathfrak{p}}} % a prime ideal
\newcommand{\qI}{{\mathfrak{q}}} % a prime ideal
\newcommand{\mI}{{\mathfrak{m}}} % a maximal ideal
\newcommand{\nil}{\mathfrak{N}} % nilradical
\newcommand{\jac}{\mathfrak{R}} % Jacobson radical

% Polynomials
\newcommand{\cont}{\text{cont}} % content (of a polynomial)

% Modules
\newcommand{\Ann}{{\text{Ann}}} % annihilator

% Fields and Galois Theory
\newcommand{\Irr}{{\text{Irr}}} % irreducible polynomial

% Algebraic Number Theory
\newcommand{\disc}{{\text{disc}}} % discriminant

% Algebraic Geometry
\newcommand{\Spec}{\text{Spec}} % prime spectrum of a ring

% misc
\newcommand{\ol}{\overline}
\begin{document}

\maketitle

Let $f(z), g(z)$ be modular forms and let $A(n), B(n)$ be their Fourier
coeficients. Define the Eisenstein series for $SL(2, \Z)$ by
\[E(z, s) = \pi^{-s} \Gamma(s) \zeta(2s) \sum_{\ol{\Gamma_\infty}
\backslash \ol{\Gamma(1)}} \img (\gamma(z))^s\] Also, define
\[\Lambda(s) = 4^{-s-k+1}\pi^{-2s-k+1}\Gamma(s)\Gamma(s + k - 1)\zeta(2s) \sum_{n = 1}^\infty
A(n)B(n)n^{-s-k+1}\] We show (*)
\[\Lambda(s) = \int_{\Gamma(1) \backslash \mathcal{H}} f(z) \ol{g}(z) E(z, s)
y^k \; \frac{\dd x \; \dd y}{y^2}\] First note that for $\gamma =
\begin{pmatrix} a & b \\ c & d \end{pmatrix} \in \Gamma(1)$ we have
\[f(\gamma(z))g(\gamma(z))\img(\gamma(z))^k = (cz + d)^k(c\ol{z} + d)^k f(z)
g(z) \frac{y^k}{|cz + d|^{2k}} = f(z)g(z)y^k\] hence $f(z) \ol{g}(z) y^k$ is
invariant under fractional linear transformations. Also, $E(z, s)$ is
automorphic in $z$. Finally, it is easy to check that the differential $\frac{\dd x
\; \dd y}{y^2}$ is invariant under linear fractional transformations by checking
it for the generators $S$ and $T$ of $\Gamma(1)$. Hence, the integral is
well-defined. 

The right hand side of $(*)$ is equal to
\[\pi^{-s}\Gamma(s)\zeta(2s) \sum_{\gamma \in \ol{\Gamma_\infty} \backslash
\ol{\Gamma(1)}} \int_{\ol{\Gamma(1)} \backslash \mathcal{H}} f(\gamma(z))\ol{g}(\gamma(z))
\img(\gamma(z))^{s + k} \; \frac{\dd x \; \dd y}{y^2}\]
\[= \pi^{-s}\Gamma(s) \zeta(2s) \int_{\Gamma_\infty \backslash \mathcal{H}}
f(z)\ol{g}(z)y^{s + k} \; \frac{\dd x \; \dd y}{y^2}\]
Now,
\[\int_{\Gamma_\infty \backslash \mathcal{H}} f(z) \ol{g}(z) y^{s + k} \; \frac{\dd x \; \dd y}{y^2}\]
\[= \sum_{n = 1}^\infty \sum_{m = 1}^\infty A(n)\ol{B}(m) \int_{\Gamma_\infty \backslash \mathcal{H}} e^{2\pi i (n - m) x} e^{-2 \pi (n + m)y} y^{s + k} \; \frac{\dd x \; \dd y}{y^2}\]
\[= \sum_{n = 1}^\infty \sum_{m = 1}^\infty A(n)\ol{B}(m) \int_0^\infty \int_0^1 e^{2\pi i (n - m)x}e^{-2\pi(n + m)y} y^{s + k - 1} \; \dd x \; \frac{\dd y}{y}\]
\[= \sum_{n = 1}^\infty A(n)\ol{B}(n) \int_0^\infty e^{-4\pi n y}y^{s + k - 1} \frac{\dd y}{y}\]
Making the substitution $y = \frac{t}{4\pi n}$ this is
\[= (4\pi)^{-s - k + 1} \Gamma(s+k-1) \sum_{n = 1}^\infty A(n)\ol{B}(n) n^{-s-k+1}\]
Finally, $B(n) = \ol{B}(n)$ because the $B(n)$ are eigenvalues of the Hecke operators, which are self adjoint. This proves $(*)$.
\end{document}