\documentclass[12pt, letterpaper, twoside]
{article}

\usepackage{amsmath}
\usepackage{amsfonts}
\usepackage{amssymb}
\usepackage{mathrsfs}
\usepackage[shortlabels]{enumitem}
\usepackage{tikz-cd}
\usepackage{accents}

\title{L-functions for GL(2)}
\author{Nick Pilotti}

% Categories
\newcommand{\ra}{\rightarrow}
\newcommand{\xra}{\xrightarrow} % labeled arrow
\newcommand{\Id}{{\text{Id}}} % the identity

% Sets
\newcommand{\sub}{{\; \subseteq \;}} % subset relation
\newcommand{\super}{{\; \supseteq \;}} % superset relation
\newcommand{\pow}{{\mathcal{P}}} % power set

% Combinatorics
\def\multiset#1#2{\ensuremath{\left(\kern-.3em\left(\genfrac{}{}{0pt}{}{#1}{#2}\right)\kern-.3em\right)}}

% Number systems
\newcommand{\N}{{\mathbb N}} % natural numbers
\newcommand{\Z}{{\mathbb Z}} % integers
\newcommand{\Q}{{\mathbb Q}} % rationals
\newcommand{\R}{{\mathbb R}} % reals
\newcommand{\C}{{\mathbb C}} % complex numbers
\newcommand{\F}{{\mathbb F}} % a finite field

% Topology
\newcommand{\Bd}{\text{Bd}} % boundary
\newcommand{\Int}{\text{Int}} % interior

% Algebraic Topology
\newcommand*\sq{\mathbin{\vcenter{\hbox{\rule{.3ex}{.3ex}}}}} % concatenation

% Geometry
\newcommand{\Proj}{\mathbb{P}} % projective space

% Metric Spaces
\newcommand{\D}{\mathbb{D}} % unit disk
\newcommand{\B}{\mathbb{B}} % unit ball

% Measure Theory
\newcommand{\meas}{\mathcal{M}} % measurable sets
\newcommand{\borel}{\mathcal{B}} % borel sets

% Analysis
\newcommand{\dd}[1]{\mathrm{d}#1} % differential
\newcommand{\del}{\partial} % partial derivative
\newcommand{\supp}{\text{supp} \:} % support of a function
\newcommand{\BV}{\text{BV}} % bounded variation

\newlength{\dhatheight}
\newcommand{\doublehat}[1]{%
    \settoheight{\dhatheight}{\ensuremath{\hat{#1}}}%
    \addtolength{\dhatheight}{-0.35ex}%
    \hat{\vphantom{\rule{1pt}{\dhatheight}}%
    \smash{\hat{#1}}}} % Fourier transform

% Complex Analysis
\newcommand{\res}{\text{res}} % residue

% Functional Analysis
\newcommand{\Xcal}{\mathcal{X}} % normed vector space
\newcommand{\Ycal}{\mathcal{Y}} % normed vector space
\newcommand{\Hcal}{\mathcal{H}} % Hilbert space

% General Algebra
\newcommand{\Ker}{{\text{Ker}}} % kernel
\newcommand{\Coker}{{\text{Coker}}} % cokernel
\newcommand{\coker}{{\text{coker}}} % cokernel
\newcommand{\Img}{{\text{Im}}} % image
\newcommand{\img}{{\text{im}}} % image
\newcommand{\Hom}{\text{Hom}} % the hom functor
\newcommand{\End}{{\text{End}}} % endomorphisms
\newcommand{\Aut}{{\text{Aut}}} % automorphisms

% Group Theory
\newcommand{\Orb}{\text{Orb}} % orbit
\newcommand{\Stab}{{\text{Stab}}} % stabilizer
\newcommand{\nor}{{\trianglelefteq} \; } % normal subgroup
\newcommand{\Cent}{\text{Cent}} % center
\newcommand{\Syl}{{\text{Syl}}} % sylow subgroups

% Representation theory
\newcommand{\Ind}{\text{Ind}} % induced representation
\newcommand{\Res}{\text{Res}} % restriction

% Specific groups
\newcommand{\Sym}{{\text{Sym}}} % symmetric group
\newcommand{\GL}{{\text{GL}}} % general linear group
\newcommand{\PGL}{{\text{PGL}}} % projective general linear group
\newcommand{\SL}{{\text{SL}}} % special linear group
\newcommand{\PSL}{{\text{PSL}}} % projective special linear group
\newcommand{\OR}{\text{O}} % orthogonal group
\newcommand{\SO}{\text{SO}} % special orthogonal group
\newcommand{\UN}{\text{U}} % unitary group
\newcommand{\SU}{\text{SU}} % special unitary group

% Rings
\newcommand{\cha}{\text{char}} % characteristic
\newcommand{\aI}{{\mathfrak{a}}} % an ideal
\newcommand{\bI}{{\mathfrak{b}}} % an ideal
\newcommand{\cI}{{\mathfrak{c}}} % an ideal
\newcommand{\sI}{{\mathfrak{s}}} % an ideal
\newcommand{\nI}{\mathfrak{n}} % an ideal
\newcommand{\pI}{{\mathfrak{p}}} % a prime ideal
\newcommand{\qI}{{\mathfrak{q}}} % a prime ideal
\newcommand{\mI}{{\mathfrak{m}}} % a maximal ideal
\newcommand{\nil}{\mathfrak{N}} % nilradical
\newcommand{\jac}{\mathfrak{R}} % Jacobson radical
\newcommand{\rad}{\text{rad}} % radical
\newcommand{\oI}{\mathfrak{o}} % integral elements

% Lie Algebras
\newcommand{\gI}{\mathfrak{g}} % a Lie algebra
\newcommand{\hI}{\mathfrak{h}} % a Lie algebra
\newcommand{\Gl}{\mathfrak{gl}} % general linear
\newcommand{\Sl}{\mathfrak{sl}} % special linear
\newcommand{\Or}{\mathfrak{o}} % orthgonal
\newcommand{\So}{\mathfrak{so}} % special orthogonal
\newcommand{\Un}{\mathfrak{u}} % unitary
\newcommand{\Su}{\mathfrak{su}} % special unitary
\newcommand{\ad}{\text{ad}} % adjoint

% Polynomials
\newcommand{\cont}{\text{cont}} % content (of a polynomial)

% Modules
\newcommand{\Ann}{{\text{Ann}}} % annihilator

% Linear Algebra
\newcommand{\Mat}{{\text{Mat}}} % matrices
\newcommand{\Tr}{{\text{Tr}}} % trace
\newcommand{\diag}{\text{diag}} % diagonal matrix
\newcommand{\Span}{\text{span}} % span

% Fields and Galois Theory
\newcommand{\K}{\mathbb{K}} % a field
\newcommand{\Irr}{{\text{Irr}}} % irreducible polynomial

% Algebraic Number Theory
\newcommand{\disc}{{\text{disc}}} % discriminant

% Algebraic Geometry
\newcommand{\Spec}{\text{Spec}} % prime spectrum of a ring

% misc
\newcommand{\ol}{\overline}
\begin{document}

\maketitle

\textbf{The case of modular forms.} Before discussing $L$-functions for
$\GL(2)$, let us first recall $L$-functions for cusp forms, so that we can draw
analogies between the two theories. Let $f$ be a cusp form of level $1$ and
weight $k$ with Fourier expansion
\[f(z) = \sum a_n e^{2\pi i n z}.\] There is an associated $L$-function
\[L(s, f) = \sum a_n n^{-s}\] which can be recognized as a Mellin transform of
$f$
\[\Lambda(s, f) = (2\pi)^{-s}\Gamma(s)L(s, f) = \int_0^{\infty} f(iy)y^s \;
\dd^* y.\]

This is seen by substituting the Fourier expansion for $f$,
\[\Lambda(s, f) = \int_0^\infty f(iy)y^s \; d^* y = \int_0^\infty \sum
a_ne^{-2\pi ny} y^s \; \dd^* s.\] Swapping the integral and sum and making the
substitution $2\pi n y \mapsto y$, we get
\[\Lambda(s, f) = (2\pi)^{-s}\Gamma(s)\sum a_nn^{-s}\] Convergence of the
integral and the validity of these operations follows from the ``moderate
growth'' properties of $f$. In case $f$ is a normalized Hecke eigenform, we also
have an Euler product
\[\Lambda(s, f) = (2\pi)^{-s}\Gamma(s) \prod_p \frac{1}{1 - a_pp^{-s} + p^{k - 1
- 2s}}.\] The functional equation for $\Lambda(s, f)$ follows from the
functional equation $f(iy) = (-1)^{k/2}y^{-k}f(i/y)$ in the following way
\[\Lambda(s, f) = \int_0^\infty f(iy)y^{s} \; \dd^* s = (-1)^{k/2} \int_0^\infty
f(i/y) y^{s - k} \; \dd^* s.\] The substitution $y \mapsto y^{-1}$ gives
\[\Lambda(s, f) = (-1)^{k/2}\Lambda(k - s, f).\]

In fact, the functional equation for $f$ is \textit{equivalent} to the
functional equation for $\Lambda$, in a sense which is made precise by the
\textit{converse theorem}:

Let $a_n$ be a sequence of complex numbers such that $|a_n| = O(n^k)$ for some
real number $k$. Let $f(z) = \sum a_n e^{2\pi i n z}$. If $\Lambda(s, f)$ has
analytic continuation to all $s$, is bounded in every vertical strip $\sigma_1
\leq \text{re}(s) \leq \sigma_2$, and satisfies $\Lambda(s, f) =
(-1)^{k/2}\Lambda(k - s, f)$, then $f$ is a cusp form of level $1$ and weight
$k$. If the Euler product is valid, $f$ is a normalized Hecke eigenform.

All of the above result can be generalized to the case of ``twisting'' by a
multiplicative character $\chi$.

\textbf{Local Whittaker models and local multiplicity one.} The analogue of
Fourier expansion in the theory of automorphic forms comes from Whittaker
models. Let $F$ be a non-Archimedean local field, $\psi$ a nontrivial additive
character of $F$, and let $(\pi, V)$ be an irreducible admissible reprensetation
of $\GL(2, F)$. There exists at most one space $\mathcal{W}$ of functions in
$\GL(2, F)$ such that if $W \in \mathcal{W}$, then
\[W \bigg( \begin{pmatrix} 1 & x \\ 0 & 1 \end{pmatrix} g\bigg) = \psi(x)W(g)\,
\; x \in F, g \in \GL(2, F)\] and such that $W$ is closed under right
translation by elements of $\GL(2, F)$, and the resulting reprensetation of
$\GL(2, F)$ is isomorphic of $\pi$. Such a space of function $\mathcal{W}$ is
called a \textit{Whittaker model} for $(\pi, V)$. There is a similar definition
for Archimedean local fields, which I will omit. The local multiplicity one
theorem holds in the Archimedean case as well.

\textbf{Whittaker models.} Let $F$ be a global field and $A$ its adele ring. We
fix $\psi$ a nontrivial character of $A/F$. $\psi$ decomposes into local
characters $\psi_v$ (c.f. Tate's thesis, section $4.1.$). Let $\pi$ be an
irreducible admissible representation of $\GL(2, A)$. A \textit{Whittaker model}
of $\pi$ with respect to the nontrivial character $\psi$ is a space of functions
of $\GL(2, A)$ which are

$(i)$ smooth,

$(ii)$ $K$-finite,

$(iii)$ of moderate growth,

$(iv)$ and satisfying
\[W \bigg( \begin{pmatrix} 1 & x \\ 0 & 1 \end{pmatrix} g \bigg) = \psi(x)
W(g)\] for all $x \in A$.

\textbf{Multiplicity one.} $(\pi, V)$ has a Whittaker model $\mathcal{W}$ with
respect to $\psi$ if and only if each $(\pi_v, V_v)$ has a Whittaker model $W_v$
with respect to the character $\psi_v$ of $F_v$. If this is the case, the
$\mathcal{W}$ is unique and consists of all finite linear combinations of
functions of the form $W(g) = \prod_v W_v(g_v)$ where $W_v \in \mathcal{W}_v$,
and $W = W_v^\circ$ for almost all $v$, where $W_v^\circ$ spherical elements of
$\mathcal{W}_v$, normalized so $W_v^\circ(k_v) = 1$ for $k_v \in \GL(2, \oI_v)$.

\textbf{Existence of Whittaker models for automorphic cuspidcal
representations.} Now suppose that $\pi$ is an automorphic cuspidal
representation. If $\phi \in V$ and $g \in \GL(2, A)$, let
\[W_\phi(g) = \int_{A/F} \phi \bigg( \begin{pmatrix} 1 & x \\ 0 & 1
\end{pmatrix} g\bigg) \psi(-x) \; \dd x.\] Then the space $\mathcal{W}$ of
functions $W_\phi$ is a Whittaker model for $\pi$. We have the ``Fourier
expansion''
\[\phi(g) = \sum_{\alpha \in F^\times} W_\phi\bigg( \begin{pmatrix} \alpha & 0
\\ 0 & 1 \end{pmatrix} g\bigg).\]

\textbf{The global zeta integral.} $Z$ is analogous to $\Lambda$ from the case
of modular forms. It is the ``completed global $L$-function.'' For $\phi \in V$,
\[\phi \begin{pmatrix} y & 0 \\ 0 & 1 \end{pmatrix}\] is rapidly decreasing as
$|y| \ra \infty$. That is, for any $N > 0$ there exists a constant $C_N$ such
that
\[\phi \begin{pmatrix} y & 0 \\ 0 & 1 \end{pmatrix} < C_N|y|^{-N}.\]

Hence,
\[Z(s, \phi) = \int_{A^\times/F^\times} \phi \begin{pmatrix} y & 0 \\ 0 & 1
\end{pmatrix} |y|^{s - 1/2} \; \dd^* y\] is absolutely convergent for all values
of $s$. By the Fourier expansion, we have
\[Z(s, \phi) = \int_{A^\times / F^\times} \sum_{\alpha \in F^\times }
W_\phi\begin{pmatrix}\alpha y & 0 \\ 0 & 1 \end{pmatrix} |y|^{s - 1/2} \; \dd^*
y = \int_{A^\times} W_\phi \begin{pmatrix} y & 0 \\ 0 & 1 \end{pmatrix} |y|^{s -
1/2} \; \dd^* y\] provided this integral is absolutely convergent. Absolute
converges occurs when $\text{re}(s) > 3/2$. We may write $W(g) = \prod_v
W_v(g_v)$ and suppose that the vector $\phi$ corresponds to a pure tensor
$\otimes_v \phi_v$. $Z$ decomposes as an Euler product
\[Z(s, \phi) = \prod_v Z_v(s, W_v),\] where
\[Z_v(s, W_v) = \int_{F_v^\times} W_v \begin{pmatrix} y_v & 0 \\ 0 & 1
\end{pmatrix} |y_v|_v^{s - 1/2} \; \dd^* y_v.\] Absolute convergence of the
local factors occurs when $\text{re}(s) > 1/2$.

\textbf{The local $L$-functions.} Let $F$ be a non-Archimedean local field and
$\oI$ its ring of integers. We say that a mutliplicative character of $F$ is
unramified it is trivial on the group of units of $F$. If $\chi_1$ and $\chi_2$
are unramified multiplicative characters of $F$, their principal series
representation of $\GL(2, F)$, denoted $\pi(\chi_1, \chi_2)$, is spherical,
i.e., has a $\GL(2, \oI)$-fixed vector. Let $\alpha_1 = \chi(\varpi)$ and
$\alpha_2 = \chi_2(\varpi)$ where $\varpi$ is a uniformizer of $F$. We call
$\alpha_1$ and $\alpha_1$ the \textit{Satake parameters} of $\pi(\chi_1,
\chi_2)$. If $\pi(\chi_1, \chi_2)$ is unitary, $\alpha_1$ and $\alpha_2$ lie on
the unit circle. Let $q$ be the cardinality of the residue fied $\oI/(\varpi)$.
We call
\[L(s, \pi) = (1 - \alpha_1q^{-s})^{-1}(1 - \alpha_2q^{-s})^{-1}\] the local
$L$-function of $\pi$.

\textbf{Equality of $Z_v$ and $L_v$.} Write $\pi = \otimes_v \pi_v$ is a tensor
product of local reprensetations $(\pi_v, V_v)$. For a place $v$, $F_v$ denotes
its corresponding local field and $\oI_v$ its ring of integers for
non-Archimedean $v$. We say that a place $v$ is unramified if $v$ is
non-Archimedean, $\pi_v$ is a spherical principal series, the conductor of the
additive character $\psi_v$ is $\oI_v$, the vector $\phi_v$ is the spherical
vector in the representation, and Whittaker function $W_v$ is normalized so that
$W_v(1) = 1$. These conditions are true for almost all $v$. 

If $v$ is unramified, then for $s$ sufficiently large,
\[Z_v(s, W_v) = L_v(s, \pi_v).\]

\textbf{The (partial) global L-function.} We assume the central chracter
$\omega$ of $\pi$ is unitary. Let $S$ be a finite set of places such that if $v
\not\in S$, $\pi_v$ is spherical. Let
\[L_S(s, \pi) = \prod_{v \not\in S} L_v(s, \pi_v).\]

\textbf{Twisting.} All of the previous results hold in the more general case of
twisting by a unitary Hecke character $\xi$. For instance, we define
\[Z(s, \psi, \xi) = \int_{A^\times/F^\times} \phi \begin{pmatrix} y & 0 \\ 0 & 1
\end{pmatrix}\xi(y)|y|^{s - 1/2} \; \dd^\times y\] and all the other definitions
are modified in a similar way.

\textbf{The global functional equation.} Let
\[w_1 = \begin{pmatrix} 0 & 1 \\ -1 & 0 \end{pmatrix}.\] Because $\phi$ is
automorphic,
\[Z(s, \phi, \xi) = \int_{A^\times/F^\times} \phi \bigg( w_1 \begin{pmatrix} y &
0 \\ 0 & 1 \end{pmatrix}\bigg) |y|^{s - 1/2} \xi(y) \; \dd^* y\]
\[= \int_{A^\times/F^\times} \phi \bigg( \begin{pmatrix} 1 & 0 \\ 0 & y
\end{pmatrix} w_1\bigg) |y|^{s - 1/2} \xi(y) \; \dd^* y.\] Substituting $y^{-1}$
for $y$, the above is equal to
\[\int_{A^\times/F^\times} (\pi(w_1)\phi) \begin{pmatrix} y & 0 \\ 0 & 1
\end{pmatrix} |y|^{-s + 1/2} (\xi\omega)^{-1}(y) \; \dd^* y.\] Hence,
\[Z(s, \phi, \xi) = Z(1 - s, \pi(w_1)\phi, \xi^{-1}\omega^{-1})\]


\textbf{Local funcitonal equation.} The local zeta integral $Z_v(s, W_v)$ has
meromorphic continuation to all $s$. There exists a meromorphic function
$\gamma_v(s, \pi_v, \xi_v, \psi_v)$ such that
\[Z_v(1 - s, \pi_v(w_1)W_v, \xi_v^{-1}\omega_v^{-1}) = \gamma_v(s, \pi_v, \xi_v,
\psi_v) Z_v(s, W_v, \xi_v).\]

\textbf{Functional equation for the partial global $L$-function.} Let $\xi$ be a
Hecke character of $F$. Let $S$ be a finite set of places of $F$ containing all
the Archimedean ones such that if $v \not\in S$, then $\pi_v$ is spherical and
$\xi_v$ is nonramified and the additive character $\psi_v$ has conductor
$\oI_v$. Then
\[L_S(s, \pi, \xi) = \bigg\{ \prod_{v \in S} \gamma_v(s, \pi_v, \xi_v, \psi_v)
\bigg\} L_S(1 - s, \hat{\pi}, \xi^{-1})\] where $\hat{\pi}$ is the
contragradient representation

\textbf{A ``local converse theoem.''} Now let $F$ be a non-Archimedean local
field. Let $\pi_1$ and $\pi_2$ be irreducible admissible representations of
$\GL(2, F)$. Suppose that $\pi_1$ and $\pi_2$ have the same central
quasicharacter $\omega$ and that $\gamma(s, \pi_1, \xi, \psi) = \gamma(s, \pi_2,
\xi, \psi)$ for all characters $\xi$ of $F^\times$. Then $\pi_1 \cong \pi_2$.

\end{document}