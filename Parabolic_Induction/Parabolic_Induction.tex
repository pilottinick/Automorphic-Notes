\documentclass[12pt, letterpaper, twoside]
{article}

\usepackage{amsmath}
\usepackage{amsfonts}
\usepackage{amssymb}
\usepackage{mathrsfs}
\usepackage[shortlabels]{enumitem}
\usepackage{tikz-cd}

\title{Parabolic Induction}
\author{Nick Pilotti}

% Categories
\newcommand{\ra}{\rightarrow}
\newcommand{\xra}{\xrightarrow} % labeled arrow
\newcommand{\Id}{{\text{Id}}} % the identity

% Sets
\newcommand{\sub}{{\; \subseteq \;}} % subset relation
\newcommand{\super}{{\; \supseteq \;}} % superset relation
\newcommand{\pow}{{\mathcal{P}}} % power set

% Combinatorics
\def\multiset#1#2{\ensuremath{\left(\kern-.3em\left(\genfrac{}{}{0pt}{}{#1}{#2}\right)\kern-.3em\right)}}

% Number systems
\newcommand{\N}{{\mathbb N}} % natural numbers
\newcommand{\Z}{{\mathbb Z}} % integers
\newcommand{\Q}{{\mathbb Q}} % rationals
\newcommand{\R}{{\mathbb R}} % reals
\newcommand{\C}{{\mathbb C}} % complex numbers
\newcommand{\F}{{\mathbb F}} % a finite field

% Topology
\newcommand{\Bd}{\text{Bd}} % boundary
\newcommand{\Int}{\text{Int}} % interior

% Algebraic Topology
\newcommand*\sq{\mathbin{\vcenter{\hbox{\rule{.3ex}{.3ex}}}}} % concatenation

% Geometry
\newcommand{\Proj}{\mathbb{P}} % projective space

% Metric Spaces
\newcommand{\D}{\mathbb{D}} % unit disk
\newcommand{\B}{\mathbb{B}} % unit ball

% Measure Theory
\newcommand{\meas}{\mathcal{M}} % measurable sets
\newcommand{\borel}{\mathcal{B}} % borel sets

% Analysis
\newcommand{\dd}[1]{\mathrm{d}#1} % differential
\newcommand{\del}{\partial} % partial derivative
\newcommand{\supp}{\text{supp} \:} % support of a function
\newcommand{\BV}{\text{BV}} % bounded variation

% Complex Analysis
\newcommand{\res}{\text{res}} % residue

% Functional Analysis
\newcommand{\Xcal}{\mathcal{X}} % normed vector space
\newcommand{\Ycal}{\mathcal{Y}} % normed vector space
\newcommand{\Hcal}{\mathcal{H}} % Hilbert space

% General Algebra
\newcommand{\Ker}{{\text{Ker}}} % kernel
\newcommand{\Coker}{{\text{Coker}}} % cokernel
\newcommand{\coker}{{\text{coker}}} % cokernel
\newcommand{\Img}{{\text{Im}}} % image
\newcommand{\img}{{\text{im}}} % image
\newcommand{\Hom}{\text{Hom}} % the hom functor
\newcommand{\End}{{\text{End}}} % endomorphisms
\newcommand{\Aut}{{\text{Aut}}} % automorphisms

% Group Theory
\newcommand{\Orb}{\text{Orb}} % orbit
\newcommand{\Stab}{{\text{Stab}}} % stabilizer
\newcommand{\nor}{{\trianglelefteq} \; } % normal subgroup
\newcommand{\Cent}{\text{Cent}} % center
\newcommand{\Syl}{{\text{Syl}}} % sylow subgroups

% Representation theory
\newcommand{\Ind}{\text{Ind}} % induced representation
\newcommand{\Res}{\text{Res}} % restriction

% Specific groups
\newcommand{\Sym}{{\text{Sym}}} % symmetric group
\newcommand{\GL}{{\text{GL}}} % general linear group
\newcommand{\PGL}{{\text{PGL}}} % projective general linear group
\newcommand{\SL}{{\text{SL}}} % special linear group
\newcommand{\PSL}{{\text{PSL}}} % projective special linear group
\newcommand{\OR}{\text{O}} % orthogonal group
\newcommand{\SO}{\text{SO}} % special orthogonal group
\newcommand{\UN}{\text{U}} % unitary group
\newcommand{\SU}{\text{SU}} % special unitary group

% Rings
\newcommand{\cha}{\text{char}} % characteristic
\newcommand{\aI}{{\mathfrak{a}}} % an ideal
\newcommand{\bI}{{\mathfrak{b}}} % an ideal
\newcommand{\cI}{{\mathfrak{c}}} % an ideal
\newcommand{\sI}{{\mathfrak{s}}} % an ideal
\newcommand{\nI}{\mathfrak{n}} % an ideal
\newcommand{\pI}{{\mathfrak{p}}} % a prime ideal
\newcommand{\qI}{{\mathfrak{q}}} % a prime ideal
\newcommand{\mI}{{\mathfrak{m}}} % a maximal ideal
\newcommand{\nil}{\mathfrak{N}} % nilradical
\newcommand{\jac}{\mathfrak{R}} % Jacobson radical
\newcommand{\rad}{\text{rad}} % radical

% Lie Algebras
\newcommand{\gI}{\mathfrak{g}} % a Lie algebra
\newcommand{\hI}{\mathfrak{h}} % a Lie algebra
\newcommand{\Gl}{\mathfrak{gl}} % general linear
\newcommand{\Sl}{\mathfrak{sl}} % special linear
\newcommand{\Or}{\mathfrak{o}} % orthgonal
\newcommand{\So}{\mathfrak{so}} % special orthogonal
\newcommand{\Un}{\mathfrak{u}} % unitary
\newcommand{\Su}{\mathfrak{su}} % special unitary
\newcommand{\ad}{\text{ad}} % adjoint

% Polynomials
\newcommand{\cont}{\text{cont}} % content (of a polynomial)

% Modules
\newcommand{\Ann}{{\text{Ann}}} % annihilator

% Linear Algebra
\newcommand{\Mat}{{\text{Mat}}} % matrices
\newcommand{\Tr}{{\text{Tr}}} % trace
\newcommand{\diag}{\text{diag}} % diagonal matrix
\newcommand{\Span}{\text{span}} % span

% Fields and Galois Theory
\newcommand{\K}{\mathbb{K}} % a field
\newcommand{\Irr}{{\text{Irr}}} % irreducible polynomial

% Algebraic Number Theory
\newcommand{\disc}{{\text{disc}}} % discriminant

% Algebraic Geometry
\newcommand{\Spec}{\text{Spec}} % prime spectrum of a ring

% misc
\newcommand{\ol}{\overline}
\begin{document}

\maketitle

We generalize the construction of representations in part $2$ and part $4$ of
Bump, and recognize them both as cases of parabolic induction. 

There are analogies between the Archimedean representation theory and the
non-Archimedean representation theory in the study of automorphic
representations. The Archimedean theory studies representations of $G_\infty =:
\GL(2, \R)$ or $G_\infty^+ =: \GL(2, \R)^+$ and the non-Archimedean theory
studies representations of $G_p =: \GL(2, \Q_p)$ for a prime $p$. Let $k$ be the
local field over which $G$ is defined.

\textbf{Maximal compact subgroup.} There exist \textit{maximal compact
subgroups} $K_\infty := \OR(2) \leq G_\infty$, $K_\infty^+ := \SO(2) \leq
G_\infty^+$ and $K_p := \GL(2, \Z_p) \leq G_p$. The maximal compact subgroup of
$G$ will be denoted by $K$.

\textbf{Group decompositions.} Let $B \leq G$ denote the Borel subgroup of upper
triangular matrices. The \textit{Iwasawa decomposition} is expressed by the
equation
\[G = BK.\] Let $N \leq G$ denote the subgroup of upper triangular unipotent
matrices and $T \leq G$ the subgroup of diagonal matrices. Note that
\[B = TN\] and this product is semidirect, i.e., $T$ is in the normalizer of $N$
and $T \cap N = 1$. $T$ is called the \textit{Levi subgroup} and is also an
example of a \textit{split tours}. $N$ is called the \textit{unipotent radical}.

Let $W = \bigg\{ \begin{pmatrix} 1 & 0 \\ 0 & 1 \end{pmatrix}, \begin{pmatrix} 0
& 1 \\ -1 & 0 \end{pmatrix}\bigg\}$. $W$ is called the Weyl group (though it is
not a subgroup of $G$, it is in the quotient of $G$ by scalars). The
\textit{Bruhat decomposition} is expressed by the equation
\[G = \cup_{w \in W} BwB\] and this union is disjoint.

\textbf{Topological groups and Haar measure.} $G$ is always a Hausdorff locally
compact topological group. $\GL_\infty$ and $\GL_\infty^+$ are \textit{Lie
groups}. $\GL_p$ is \textit{totally disconnected}, i.e., every neighborhood of
the identity contains an open compact subgroup. 

There exist left and right-invariant measures on $G$, called the left and right
\textit{Haar measures}, and denoted by $\dd_Lg$ and $\dd_Rg$ respectively. They
are unique up to scalar multiple. For a Hausdorff locally compact group $H$, the
\textit{modular quasicharacter} $\delta_H$ is defined by $\dd_R h = \delta_H
(h)\dd_L h$. We say that $H$ is \textit{unimodular} if $\delta_H = 1$, i.e., the
left and right Haar measures coincide.

In any case, $G$ and $K$ are unimodular. We choose $\dd g$ so that $\int_K \;
\dd g = 1$. The modular quasicharacter of $B$ is $\delta_B
\begin{pmatrix} a & c \\ 0 & b \end{pmatrix} = |a/b|$. 

\textbf{Smooth functions.} Let $H$ be a subgroup of $G$. If $G$ is Archimedean,
we say that a complex-valued function on $H$ is \textit{smooth} if it is smooth
as a map between manifolds. If $G$ is non-Archimedean, we say that a
complex-valued function on $H$ is \textit{smooth} if it is locally constant. Let
$C^\infty(H)$ denote the set of smooth complex-valued maps on $H$. Let
$C_c^\infty(H)$ denote the set of smooth complex-valued maps on $H$ with compact
support.

\textbf{Representations.} Let $(\pi, V)$ be a representation of $G$. If $G$ is
Archimedean, we will study $\pi$ when it is a (not necessary unitary) Hilbert
space representation or when it is a $(\gI, K)$\textit{-module}, where $\gI$ is
the Lie algebra of $G$ (c.f. Bump $2.4$). Equivalence of $(\gI, K)$-modules is
called infinitesimal equivalence. If $G$ is non-Archimedean, we will study $\pi$
when it is \textit{smooth}. That is, for each $v \in V$, the stabilizer of $v$
in $G$ is open.

\textbf{Direct sum decomposition and admissible representations.} Let $(\pi, V)$
be representation of $G$ which is a Hilbert space representation if $G$ is
Archimedean and smooth if $G$ is non-Archimedean. Let $\hat{K}_\infty$ (resp.
$\hat{K}_\infty^+$) be the set of equivalence classes of finite-dimensional
unitarizable irreducible representations of $K_\infty$ (resp. $K_\infty^+$). Let
$\hat{K}_p$ be the set of equivalence classes of finite-dimensional irreducible
representations of $K_p$ whose kernel is open (hence of finite index). Denote
this object generally by $\hat{K}$. If $\rho \in \hat{K}$, let $V(\rho)$ be the
sum of all $K$-invariant subspaces of $V$ that are isomorphic as $K$-modules to
$\rho$. Then,
\[V = \oplus_{\rho \in \hat{K}} V(\rho)\] If $G$ is Archimedean, this is a
Hilbert space direct sum of $K$-modules. If $G$ is non-Archimedean, this is an
algebraic direct sum of $K$-modules.

We say that $\pi$ is \textit{admissible} if for each $\rho \in \hat{K}$,
$V(\rho)$ is finite dimensional. If $G$ is Archimedean, an irreducible unitary
representation is always admissible. If $G$ is non-Archimedean, then $\pi$ is
admissible if and only if for any open subgroup $U \leq G$, the space $V^U$ of
vectors stabilized by $U$ is finite dimensional.

\textbf{Multiplicative characters.} A \textit{quasicharacter} of a group $H$ is
a homomorphism $\chi$ from $H$ to the multiplicative group $\C^\times$ of
complex numbers. We say that $\chi$ is a \textit{character} (or \textit{unitary
character}) if $|\chi| = 1$. 

Tate's thesis (section $2.3$) provides a useful description of the
quasicharacters on $k^\times$. Let $u$ denote the kernel of the absolute value
map $| \cdot | : k \ra \R_{\geq 0}$ (so if $F = \R$, $U = \{\pm 1\}$ and if $F =
\Q_p$, $U = \{ x \in \Z_p : p \nmid x$\}). We call a quasicharacter
\textit{unramified} if it is trivial on $u$. The unramified quasicharacters are
the maps of the form $c(\alpha) = |\alpha|^s$, where $s$ is any complex number,
$s$ is determined by $c$ if $k$ is Archimedean, and $s$ is determined mod $2\pi
i$. The quasicharacters are maps of the form $c(\alpha) =
\tilde{c}(\tilde{\alpha})|\alpha|^s$, where $\tilde{c}$ is any character of $u$,
$\tilde{c}$ is uniquely determined by $c$, and $s$ is determined as in the case
of unramified quasicharacters. In particular, the quasicharacters on $\R$ are
maps of the form $c(\alpha) = \text{sgn}(\alpha)^\epsilon|\alpha|^s$, where
$\epsilon$ is an equivalence class of $\Z$ modulo $2$.

If $\chi_1$ and $\chi_2$ are quasicharacters of $k$, then
\[\chi \begin{pmatrix} a & c \\ 0 & b \end{pmatrix} = \chi_1(a)\chi_2(c)\]
defines a quasicharacter of $B$.

\textbf{Induced Representation for a quasicharacter of B.} Let $\chi$ be a
quasicharacter of $B$. Let $V^\infty$ be the set of $f \in C^\infty(G)$ such
that
\[f(bg) = \delta_G(b)^{-1/2}\delta_B(b)^{1/2}\chi(b)f(g)\] for all $b \in B$ and
$g \in G$. By the Iwasawa decomposition, functions $f \in V^\infty$ are uniquely
determined by their restriction to $K$. $G$ acts on $V^\infty$ by right
translation, i.e.,
\[(\pi(g)f)(x) = f(xg)\]

If $G$ is Archimedean, we define a Hermitian innner product on $V^\infty$ by
\[\langle f_1, f_2 \rangle = \int_K f_1(k)\ol{f_2(k_\theta)} \; \dd k\] Let $V$
be the Hilbert space completion of $V^\infty$ in $C^\infty(G)$. The action of
$G$ on $V^\infty$ can be extended to $V$, and $V^\infty$ is the space of smooth
vectors for $V$ (c.f. Bump $2.4$).

If $G$ is non-Archimedean, let $V = V^\infty$. Let $V_c = V \cap C_c^\infty(G)$
be the subspace of $V$ which consists of compactly supported smooth functions.
$V_c$ is a subrepresentation of $V$.

$V$ is called the \textit{induced representation} of $\chi$ in $G$, also denoted
$\Ind_B^G \chi$. If $G$ is non-Archimedean, we call $V_c$ the \textit{compact
induced representation}.

\textbf{Classification of $(\gI, K)$-modules for $\GL(2, \R)^+$.} For the
following, $G = \GL_\infty^+$. Note that
\[\kappa_\theta = \begin{pmatrix} \cos(\theta) & \sin(\theta) \\ -\sin(\theta) &
\cos(\theta) \end{pmatrix}, \; \theta \in [0, 2\pi)\] parameterizes elements of
$K$. Since $K \cong \R/\Z$, the irreducible unitary representations of $K$ are
one dimensional and are parameterized by integers. They are the characters
$\sigma_k(\kappa_\theta) = e^{ik\theta}$ with $k \in \Z$. We will denote the
isotypic component $V(\sigma_k)$ as simply $V(k)$. Thus,
\[V = \oplus_{k \in \Z} V(k)\] Each $V(k)$ is one dimensional. Let $\Sigma$ be
the set of all integers such that $V(k) \neq 0$. We call $\Sigma$ the
\textit{set of $K$-types of $V$}. If $V$ is an irreducible admissible $(\gI,
K)$-module, $\Sigma$ is either all even or all odd. We say the \textit{parity}
of $V$ is even or odd accordingly.

Let $s_1$ and $s_2$ be complex numbers and $\epsilon$ a equivalence class of
$\Z$ modulo $2$. Let $\chi_1$ and $\chi_2$ be the quasicharacters of $\R^\times$
given by $\chi_1(\alpha) = \text{sgn}(\alpha)^\epsilon|\alpha|^{s_1}$ and
$\chi_2(\alpha) = |\alpha|^{s_2}$. Let $\chi$ be the quasicharacter on $B$ given
by $\chi_1$ and $\chi_2$. Let $V$ be the induced representation of $\chi$ in
$G$. $V^\infty$ consists of $f \in C^\infty(G)$ such that
\[f \bigg( \begin{pmatrix} a & c \\ 0 & b \end{pmatrix} g\bigg) = a^{s_1 +
1/2}b^{s_2 - 1/2}f(g), \; g \in G, \; a, c > 0\] and
\[f \bigg( \begin{pmatrix} -1 & 0 \\ 0 & -1 \end{pmatrix} g\bigg) =
(-1)^\epsilon f(g), \; g \in G.\]

The restriction of $f$ to $K$ can be any smooth function, subject to the
condition
\[f(\kappa_{\theta + \pi}) = (-1)^\epsilon f(\kappa_\theta)\] for all $\theta$.

Now let,
\[R = \frac{1}{2} \begin{pmatrix} 1 & i \\ i & -1 \end{pmatrix}, \; L =
\frac{1}{2} \begin{pmatrix} 1 & -i \\ -i & -1 \end{pmatrix}, \; H = -i
\begin{pmatrix} 0 & 1 \\ -1 & 0 \end{pmatrix}, \; Z = \begin{pmatrix} 1 & 0 \\ 0
& 1 \end{pmatrix} \] be elements of the Lie algebra $\gI_\C = \Gl(n, \C)$, which
is the complexification of $\gI$. Define $\Delta$ by
\[-4\Delta = H^2 + 2RL + 2LR.\] If $\ell \equiv \epsilon$ mod $2$, there exists
a unique element $f_\ell$ of $V$ whose restriction satisfies
\[f_\ell(\kappa_\theta) = e^{i\ell \theta}.\] We have
\[H f_\ell = \ell f_\ell, R f_\ell = (s + 1/2) f_{\ell + 2}, \ Lf_{\ell} = (s -
1/2) f_{\ell - 2}\] and
\[\Delta f_\ell = \lambda f_\ell, \; Z f_\ell = \mu f_\ell\] where
\[\lambda = s(1 - s), \; \mu = s_1 + s_2, \; s = \frac{1}{2}(s_1 - s_2 + 1)\]

Let $V_{\text{fin}}$ be space of $K$-finite vectors of $V$ (c.f. Bump $2.4$).
Then $\Delta$ and $Z$ act as scalars on $V_{\text{fin}}$, with eigenvalues
$\lambda$ and $\mu$, respectively. The set of $K$-types of $V_{\text{fin}}$
consists of all integers congruent to $\epsilon$ modulo $2$. 

$(i)$ Suppose that $s$ is not of the form $\frac{k}{2}$, where $k$ is an integer
congruent to $\epsilon$ mod $2$. Then $V_{\text{fin}}$ is irreducible. 

$(ii)$ Suppose that $s \geq \frac{1}{2}$ and that $s = \frac{k}{2}$, where $k
\geq 1$ is an integer congruent to $\epsilon$ module $2$. Then $V_{\text{fin}}$
has two irreducible invariant subspace $V_{\text{fin}}^+$ and
$V_{\text{fin}}^-$. The set of $K$-types of $V_{\text{fin}}^+$ is
\[\Sigma^+(k) = \{ \ell \in \Z | \ell \equiv k \text{ mod } 2, \ell \geq k\}\]
and the set of $K$-types of $V_{\text{fin}}^-$ is
\[\Sigma^-(k) = \{ \ell \in \Z | \ell \equiv k \text{ mod } 2, \ell \leq -k\}\]
The quotient $V_{\text{fin}}^0 = V_{\text{fin}} / (V_{\text{fin}}^+ \oplus
V_{\text{fin}}^-)$ is irreducible, unless $k = 1$, in which case it is zero. Its
set of $K$-types is
\[\Sigma^0(k) = \{ \ell \in \Z | \ell \equiv k \text{ mod } 2, -k < \ell < k\}\]

$(iii)$ Suppose that $s \leq \frac{1}{2}$ and that $s = 1 - \frac{k}{2}$, where
$k \geq 1$ is an integer congruent to $\epsilon$ mod $2$. Then $V_{\text{fin}}$
has an invariant subspace $V_\text{fin}^0$ whose set of $K$-types is
$\Sigma^0(k)$. Here $V_{\text{fin}}^0$ is irreducible, unless $k = 1$, in which
case it is zero. The quotient $V_{\text{fin}}/V_{\text{fin}}^0$ decomposes into
two irreducible invariant subspaces $V_{\text{fin}}^+$ and $V_{\text{fin}}^-$.
The set of $K$-types of $V_{\text{fin}}^\pm$ is $\Sigma^{\pm}(k)$.

There exists exactly one isomorphism class of $(\gI, K)$-module with each given
$\Sigma$. Every irreducible admissible $(\gI, K)$-module is realized in one of
the above cases.

Let $\mathcal{P}_\mu(\lambda, \epsilon)$ denote the infinitesimal equivalence
class of irreducible admissible $(\gI, K)$-modules of parity $\epsilon$ on which
$\Delta$ and $Z$ have eigenvalues values $\lambda$ and $\mu$, respectively. If
$\mu = 0$, we denote this infinitesimal equivalence class of representations by
$\mathcal{P}(\lambda, \epsilon)$. These representations are called
\textit{principal series representations}, and are exactly those representations
which are equivalent to the irreducible admissible representation in case $(i)$.
We call a representations which is equivalent to one of the two infinite
dimensional irreducible admissible representations in case $(ii)$ or $(iii)$ a
reprsentation of the \textit{discrete series}. They are denoted by
$\mathcal{D}^\pm(k)$.



\textbf{Classification of $(\gI, K)$-modules for $\GL(2, \R)$.} For the
following, $G = \GL_\infty$. Let $s_1, s_2, \epsilon,$ and $\chi$ be as in the
previous section. $\chi$ is a quasicharacter on $B^+$, the Borel subgroup of
$\GL(2, \R)^+$. Let $\epsilon_1, \epsilon_2$ be congruence classes of $\Z$ mod
$2$ such that $\epsilon_1 + \epsilon_2 \equiv \epsilon$ mod $2$. There are two
possible choices for $\epsilon_1$ and $\epsilon_2$. Let $\chi_i : \R$ be the
quasicharacters on $\R^\times$ given by $\chi_i(y) =
\text{sgn}(y)^{\epsilon_i}|y|^{s_i}$ and extend $\chi$ to all of $B$ by $\chi
\begin{pmatrix} a & c \\ 0 & b\end{pmatrix} = \chi_1(a)\chi_2(b)$.

Let $V$ be the induced representation of $\chi$ in $G$, $V^\infty$ the space of
smooth vectors, and $V_{\text{fin}}$ the space of $K$-finite vectors. The
restriction of $G$ to $G^+$ is the same representation as in the previous
paragraph (depending only on $s_1, s_2, \epsilon$).

This construction is used to show that every irreducible admissible $(\gI,
K)$-module may be realized as the space of $K$-finite vectors is an admissible
representation of $G = \GL(2, \R)$ on a Hilbert space.

$(i)$ The finite-dimensional representations are obtained by tensoring the
symmetric powers of the standard representation with the one-dimensional
representation of the form $\chi \circ \det$, where $\chi$ is a character of
$\R^\times$.

$(ii)$ If $\chi_1$ and $\chi_2$ are quasicharacters of $\R^\times$ such that
$\chi_1\chi_2^{-1}$ is not a quasicharacter of the form $y \mapsto
\text{sgn}(y)^\epsilon |y|^{k - 1}$, where $\epsilon$ is an equivalence class of
$\Z$ mod $2$ and $k$ is an integer of the same parity of $\epsilon$, then
$V_{\text{fin}}$ is irreducible. 

$(iii)$ If the hypothesis of case $(ii)$ does not hold, then
$V_{\text{fin}}/V_{\text{fin}}^0 = \mathcal{D}_\mu^+(k) \oplus
\mathcal{D}_\mu^-(k)$. The $\mathcal{D}_\mu^{\pm}(k)$ are irreducible.

Every irreducible admissible $(\gI, K)$-module is isomorphic to one of the above
cases.

\textbf{The Jacquet module, Jacquet functor, and supercuspidal representations.}
For the following, $G$ is non-Archimedean. Let $(\pi, V)$ be a smooth
representation $B$. Define $V_N$ to be the vector subspace of $V$ generated by
elements of the form $\pi(u)v - v$ for $u \in N$, $v \in V$. The quotient $J(V)
= V/V_N$ is a smooth representation of $T$ called the \textit{Jacquet module} of
$V$. This defines a functor, called the \textit{Jacquet functor}, from the
category of smooth representations of $B$ to the category of smooth
representations of $T$. This functor is exact. An irreducible admissible
representation of $G$ is called \textit{supercuspidal} if $J(V) = 0$.

\textbf{Classification of irreducible admissible non-supercuspidal
representation of $\GL_p$.}  For the following, $G$ is non-Archimedean. Let
$\chi_1$ and $\chi_2$ be quasicharacters of $F^\times$. Fix two characters
$\xi_1$ and $\xi_2$ of $u$ such that $\chi_i(y) = |y|^{s_i}\xi(y) \; (i = 1,
2)$. Let $\chi$ be the quasicharacter on $B$ given by $\chi_1$ and $\chi_2$. Let
$\mathcal{B}(\chi_1, \chi_2)$ be the induced representation of $\chi$ in $G$. 

$(i)$ Suppose we don't have $(\chi_1\chi_2^{-1})(y) = |y|^{-1}$ for all $y \in
F^\times$ or $(\chi_1\chi_2^{-1})(y) = |y|$ for all $y \in F^\times$. Then
$\mathcal{B}(\chi_1, \chi_2)$ is irreducible and $\mathcal{B}(\chi_1, \chi_2)
\cong \mathcal{B}(\chi_2, \chi_1)$. If $\mu_1$ and $\mu_2$ are quasicharacters
of $F^\times$, we have $\mathcal{B}(\chi_1, \chi_2) = \mathcal{B}(\mu_1, \mu_2)$
if and only if either $\chi_1 = \mu_1$ and $\chi_2 = \mu_2$ or $\chi_1 = \mu_2$
and $\chi_2 = \mu_1$.

$(ii)$ If $(\chi_1\chi_2^{-1})(y) = |y|^{-1}$ for all $y \in F$, then
$\mathcal{B}(\chi_1, \chi_2)$ has a one dimensional invariant subspace and the
quotient representation is irreducible. 

$(iii)$ If $(\chi_1\chi_2^{-1})(y) = |y|$, then $\mathcal{B}(\chi_1, \chi_2)$
has an irreducible invariant subspace of codimension one.

Each admissible irreducible non-supercuspidal representation of $G$ is
isomorphic to an irreducible representation covered by one of the above cases.

An representation which is isomorphic to one of the irreducible representations
covered by case $(i)$ is called a \textit{principal series represntation}. A
representaiton which is isomorphic to one of the infinite dimensional
irreducible representations covered by cases $(ii)$ or $(iii)$ is called a
\textit{special representation} or \textit{Steinberg representation}. Thus each
admissible irreducible non-supercuspidal representation of $G$ is either a
principal series representation, a special representation, or one dimensional.

\textbf{Uniqueness of the K-fixed vector.} Suppose that $G$ is Archimedean.
Assume that $\pi$ is an irreducible unitary representation of $G$ and for $\phi
\in C_c^\infty(K \backslash G/K)$, the operator $\pi(\phi)$ is a compact
operator on $V^{K}$. Then the dimension of $V^{K}$ is at most one.

Suppose that $G$ is non-Archimedean. Assume that $\pi$ is an irreducible
admissible representation of $G_p$. Then $V^K$ is at most one dimensional, and
the space of linear functionals $L$ on $V$ satisfying
\[L(\pi(k)v) = L(v), \; k \in K, v \in V\] is at most one dimensional.

TODO: \textbf{Intertwining integrals.}

\textbf{Summary.} In both theories we are dealing with Hausdorff locally compact
topological groups of matrices. We restrict our attention to those
representations which are compatible with the topology of the group in a certain
sense and which are admissible. These groups have a similar structure, in that
they both contains a maximal compact subgroup $K$ and a Borel subgroup $B$ of
upper triangular matrices. Representations are obtained by picking two
multiplicative quasicharacters $\chi_1$ and $\chi_2$ on $F^\times$ and using
them to define a quasicharacter $\chi$ on $B$. From $\chi$ we induce a
representation of $G$. The unramified parts of $\chi_1$ and $\chi_2$ determine
complex numbers $s_1$ and $s_2$, which control important properties of the
representation. It is also important to consider the case where $\chi_1$ and
$\chi_2$ are unramified. As long as the function $\chi_1\chi_2^{-1}$ is not of a
certain form, these representation are irreducible. Representations of this form
are called the principal series representations. If the representations are
reducible, they contain in their composition series one or two infinite
dimensional irreducible representations and a finite dimensional representation.
The infinite dimensional representations are called discrete series in the
Archimedean case and special in the non-Archimedean case. All irreducible
admissible representations, besides the supercuspidal ones, are obtained in this
way. Intertwining intergrals are used to defined $G$-equivariant maps between
representations.

\end{document}