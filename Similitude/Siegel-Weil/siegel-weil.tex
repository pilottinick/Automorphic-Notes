\documentclass[12pt]{article}

\usepackage{microtype}

\usepackage{amsmath}
\usepackage{amsfonts}
\usepackage{mathtools}

\usepackage{tikz-cd}

\usepackage{setspace}

\onehalfspacing

\title{Similitude Siegel-Weil Formula}
\author{Nick Pilotti}
\date{\today}

\begin{document}

\maketitle

Let $k = \mathbb{Q}$. Let $V$ be a $m$ dimensional quadratic space over $k$, and
$W$ a $2n$ dimensional symplectic space over $k$. Let $H =
\text{GO}\left(V\right)$ and let $H_{1} = \text{O}\left(V\right)$ be the kernel
of the similitude map $\lambda$. If $m$ is even, let $G' =
\text{GSp}\left(W\right)$. If $m$ is odd, let $G'$ be
$\widehat{\text{GSp}\left(W\right)}$, a certain the two-fold cover of
$\text{GSp}\left(W\right)$. For $g \in G'$, let $\lambda\left(g\right)$ be the
similitude factor of the projection of $g$ to $\text{GSp}\left(W\right)$. Let
$G$ subgroup of $g \in G'$ such that $\lambda\left(g\right) \in
\lambda\left(H\right)$. Let $G_1$ be the subgroup of $g \in G$ such that
$\lambda\left(g\right) = 1$.

Let $\mathbb{W} = V \otimes_{k} W$. $\text{GMp}\left(\mathbb{W}\right)$ is a
certain extension of $\text{GSp}\left(\mathbb{W}\right)$ by $\mathbb{C}^{1}$,
called the metaplectic cover of $\text{GSp}\left(\mathbb{W}\right)$. The typical
element of $\text{GMp}\left(\mathbb{W}\right)$ will be denoted by $\left(g,
\epsilon\right)$, where $g \in \text{GSp}\left(\mathbb{W}\right)$ and $\epsilon
\in \mathbb{C}^{1}$. $p$ is the projection from
$\text{GMp}\left(\mathbb{W}\right)$ to $\text{GSp}\left(\mathbb{W}\right)$.
There are inclusions 
%
\[H \xhookrightarrow{} \text{GSp}\left(\mathbb{W}\right), \; G
\xhookrightarrow{} \text{GSp}\left(\mathbb{W}\right).\]
%

These are inclusions are selected so that their preimages in
$\text{GMp}\left(\mathbb{W}\right)$ are nice. $p^{-1}\left(H\right)$ is trivial
as an extension of $H$. If $m$ is even, then
$p^{-1}\left(G\right)$ is trivial as an extension of $G$. If $m$ is odd, then
$p^{-1}\left(G\right)$ is the metaplectic cover of $\text{GSp}\left(Y\right)$.

Let $\psi$ be a fixed additive character of $k$ and $\omega = \omega_{\psi}$ the
standard Weil representation of $H_{1}\left(\mathbb{A}\right) \times
G_{1}\left(\mathbb{A}\right)$ on the Schwartz space
$S\left(V\left(\mathbb{A}\right)^{n}\right)$. Let 
%
\[R = \left\{\left(h, g\right) \in H \times G \; | \; \nu\left(h\right) =
\nu\left(g\right)\right\}.\]
%
There is a representation of $R\left(\mathbb{A}\right)$ on
$S\left(V\left(\mathbb{A}\right)^{n}\right)$ given by 
%
\[\omega\left(h, g\right)\varphi\left(x\right) =
\left|\nu\left(h\right)\right|^{-\frac{nm}{4}}\left(\omega\left(g_1\right)\varphi\right)\left(h^{-1}x\right).\]
%
We call $\omega$ the extended Weil representation.

The theta kernel, defined for $\left(h, g\right) \in R\left(\mathbb{A}\right)$
by 
%
\[\theta\left(h, g; \varphi\right) = \sum_{x \in V\left(k\right)^{n}}^{}
\omega\left(h, g\right)\varphi\left(x\right)\]
%
is left $R\left(k\right)$ invariant. The theta integral is defined (when
convergent) by 
%
\[I\left(g, \varphi\right) = \int_{H_{1}\left(k\right) \backslash
H_{1}\left(\mathbb{A}\right)}^{} \theta\left(h_1h, g; \varphi\right) \,
\mathrm{d}h_{1}\]
%
where $g \in G\left(\mathbb{A}\right)$, $h \in H\left(\mathbb{A}\right)$ with
$\lambda\left(h\right) = \lambda\left(g\right)$, and $\varphi \in
S\left(V\left(\mathbb{A}\right)^{n}\right)$. It does not depend on the choice of
$h$. A technical procedure is needed to define the theta integral in all cases.

$\lambda_{s}$ is a certain character of Siegel parabolic $P$ in $G$. Let
$I\left(s\right) = I_{P}^{G}\left(\lambda_{s}\right)$ be the normalized induced
representation of $G\left(\mathbb{A}\right)$. The Eisenstein series associated
to a section $\Phi_{s} \in I\left(s\right)$ is defined for
$\text{Re}\left(s\right) > 1$ by 
%
\[E\left(g, s, \Phi_{s}\right) = \sum_{\gamma \in P\left(k\right) \backslash
G\left(k\right)}^{} \Phi_{s}\left(\gamma g\right),\]
%
and the normalized Eisenstein series is 
%
\[E^{*}\left(g, s, \Phi_{s}\right) = b_{G}\left(s\right) \cdot E\left(g, s,
\Phi_{s}\right)\]
%
where $b_{G}\left(s\right)$ is a certain function of $s$.

The Siegel-Weil formula says that for a section $\Phi_{s} \in I\left(s\right)$
with $\Phi_{0} \in \Pi\left(V\right)$ so that $\Phi_{0} =
\left[\varphi\right]^{~}$ for some $\varphi \in
S\left(V\left(\mathbb{A}\right)^{n}\right),$ 
%
\[E\left(g, 0, \Phi_{s}\right) = 2 I\left(g, \varphi\right).\]
%

\end{document}
