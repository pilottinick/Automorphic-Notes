\documentclass[12pt]{article}

\usepackage{microtype}

\usepackage{amsmath}
\usepackage{amsfonts}

\usepackage{tikz-cd}

\title{Notes on ``On a conjecture of Jacquet'' by Harris and Kudla}
\author{Nick Pilotti}
\date{\today}

\begin{document}

\maketitle

\textbf{Theorem to be proved.} Let $k$ be a number field and let $\pi_{i}, \; i =
1, 2, 3$ be cuspidal automorphic representations of
$\text{GL}_{2}\left(\mathbb{A}\right)$ such that the product of their central
characters is trivial. Jacquet conjectured that the central value
$L\left(\frac{1}{2}, \pi_{1} \otimes \pi_2 \otimes \pi_{3}\right)$ of the triple
product L-function is nonzero if and only if there exists quaternion algebra
$B$ over $k$ and automorphic forms $f_{i}^{B} \in \pi_{i}^{B}$ such that the
integral 
%
\[I\left(f_{1}^{B}, f_{2}^{B}, f_{3}^{B}\right) = \int_{Z\left(\mathbb{A}\right)
B^{\times}\left(k\right) \backslash B^{\times}\left(\mathbb{A}\right)}^{}
f_1^{B}\left(b\right)f_2^{B}\left(b\right)f_3^{B}\left(b\right) \,
\mathrm{d}^{\times} b \neq 0\] 
%
where $\pi_{i}^{B}$ is the representation of $B^{\times}\left(\mathbb{A}\right)$
corresponding to $\pi_{i}$.

\textbf{Summary.} The proof comes down to the equality 
%
\[L\left(\frac{1}{2}, \pi_{1} \otimes \pi_{2} \otimes \pi_{3}\right) \cdot
Z^{*}\left(F, \Phi\right)\]
% 
\[= \sum_{r}^{} \int_{\mathbb{A}^{\times} B^{\times}\left(k\right) \backslash
B^{\times}\left(\mathbb{A}\right)}^{} I^{1, r}\left(b_1, \varphi; F\right) \,
\mathrm{d}b_{1} \cdot \int_{\mathbb{A}^{\times}B^{\times}\left(k\right)
\backslash B^{\times}\left(\mathbb{A}\right)}^{} I^{2,r}\left(b_{2}, \varphi; F\right) \, \mathrm{d}b_2\]
%
The integrals in the second expression are finite linear combinations of the
quantities $I\left(f_1^{B}, f_2^{B}, f_{3}^{B}\right)$ and every such quantity
can be obtained as an integral
$\int_{\mathbb{A}^{\times}B^{\times}\left(k\right) \backslash
B^{\times}\left(\mathbb{A}\right)}^{} I^{1, r}\left(b_{1}, \varphi; F\right) \,
\mathrm{d}b_{1}$ for some $\varphi, F$ and $r$.

Section $1$ introduces a representation of the triple product L-function in
terms of a Zeta integral and the integral of an Eisenstein series. Section $2$
shows how the integral representation should be modified by changing the Zeta
integral at the local factors.
Section $3$ constructs 

\section{The integral representation of the triple product $L$-function.}

\begin{itemize}
    
\item Let $G = \text{GSp}_{6}$ be the group of similitudes of the standard $6$
dimensional symplectic vector space over $k$, and let $P = MN$ be the Siegel
parabolic.

\item Let $K_{G} = K_{G, \infty} \cdot K_{G, f}$ be the standard maximal compact
subgroup of $G\left(\mathbb{A}\right)$.

\item $\lambda_{s}$ denotes a character of $P\left(\mathbb{A}\right)$
parameterized by $s \in \mathbb{C}$.

\item Let $I\left(s\right) = I_{P}^{G}\left(\lambda_{s}\right)$ be the
normalized induced representation of $G\left(\mathbb{A}\right)$.

\item The Eisenstein series associated to a section $\Phi_{s} \in
I\left(s\right)$ is defined for $\text{Re}\left(s\right) > 1$ by 
%
\[E\left(g, s, \Phi_{s}\right) = \sum_{\gamma \in P\left(k\right) \backslash
G\left(k\right)}^{}  \Phi_{s}\left(\gamma g\right),\]
%
and the normalized Eisenstein series is 
%
\[E^{*}\left(g, s, \Phi_{s}\right) = b_{G}\left(s\right) \cdot E\left(g, s,
\Phi_{s}\right),\]
%
where $b_{G}\left(s\right)$ is a function of $s$. These functions have
meromorphic analytic continuations to the whole $s$-plane and have no poles on
the unitary axis $\text{Re}\left(s\right) = 0$.

\item The map 
%
\[E^{*}\left(0\right) : I\left(0\right) \rightarrow \mathcal{A}\left(G\right),
\; \Phi_{0} \mapsto \left(g \mapsto E^{*}\left(g, 0, \Phi_{s}\right)\right)\]
%
gives a $\left(g_{\infty}, K_{G,\infty}\right) \times
G\left(\mathbb{A}_{f}\right)$-intertwining map from the induced representation
$I\left(0\right)$ at $s = 0$ to the space of automorphic forms on $G$ with
trivial central character.

\item Let 
%
\[\textbf{G} = \left(\text{GL}_{2} \times \text{GL}_{2} \times
\text{GL}_{2}\right)_{0}\]
%
\[= \left\{\left(g_1, g_2, g_3\right) \in \left(\text{GL}_{2}\right)^{3} |
\text{det}\left(g_1\right) = \text{det}\left(g_2\right) =
\text{det}\left(g_3\right)\right\}.\]
%
This group embeds diagonally in $G = \text{GSp}_{6}$.

\item For automorphic forms $f_{i} \in \pi_{i}, i = 1, 2, 3$, let $F = f_1
\otimes f_2 \otimes f_3$ be the corresponding function on
$G\left(\mathbb{A}\right)$. The global zeta integral is given by 
%
\[Z\left(s, F, \Phi_{s}\right) = \int_{Z_{G}G\left(k\right) \backslash
G\left(\mathbb{A}\right)}^{} E^{*}\left(g, s, \Phi_{s}\right) F\left(g\right) \, \mathrm{d} g.\]
%
\item Suppose that the automorphic forms $f_{i} \in \pi_{i}$ have factorizable
Whittaker functions $W_{i}^{\psi} = \otimes_{v} W_{i,v}^{\psi}$ and that the
section $\Phi_{s}$ is factorizable. Let $S$ be a finite set of places of $k$,
including all archimedean places, such that, for $v \not \in S$, the local data
satisfies properties $(i)$, $(ii)$, and $(iii)$ in the main text. Then 
%
\[Z\left(s, F, \Phi_{s}\right) = L^{S}\left(s + \frac{1}{2}, \pi_1 \otimes \pi_2
\otimes \pi_3\right) \cdot \prod_{v \in S} Z_{v}\left(s, W_{v}^{\psi}, \Phi_{s,
v}\right),\]
%
for local zeta integrals $Z_{v}\left(s, W_{v}^{\psi}, \Psi_{s, v}\right)$, where
$W_{v}^{\psi} = W_{1, v}^{\psi} W_{2, v}^{\psi} W_{3, v}^{\psi}$. $L^{S}\left(s,
\pi_1 \otimes \pi_2 \otimes \pi_3\right)$ is the triple product $L$-functions
with the factors for $v \in S$ omitted.

\item The local zeta integrals are defined by 
%
\[Z_{v}\left(s, W_{v}^{\psi}, \Phi_{s, v}\right) =
\int_{Z_{G}\left(k_{v}\right)M\left(k_{v}\right) \backslash
G\left(k_{v}\right)}^{} \Phi_{s, v}\left(\delta g\right)
W_{v}^{\psi}\left(g\right) \, \mathrm{d}g\] 
%
where $\delta \in G\left(k\right)$
is a representative for the open orbit of $G$ in $P \backslash G$.

\end{itemize}

\section{Local zeta integrals.}

\begin{itemize}

\item The key result is the existance of a local Euler factor. Let $\pi_{i, v},
\; i = 1, 2, 3$, be a triple of admissible irreducible representations of
$\text{GL}\left(2, k_{v}\right)$ that arise as local components at $v$ of
cuspidal automorphic representations $\pi_{i}$. The quotient 
%
\[\tilde{Z}_{v}\left(s, W_{v}^{\psi}, \Psi_{s, v}\right) = Z_{v}\left(s,
W_{v}^{\psi}, \Phi_{s, v}\right) \cdot L\left(s + \frac{1}{2}, \pi_{1, v}
\otimes \pi_{2, v} \otimes \pi_{3, v}\right)^{-1}\]
%
is entire as a function of $s$.

\item Consequently, we have the identity
%
\begin{multline}L\left(\frac{1}{2}, \pi_{1} \otimes \pi_{2} \otimes
\pi_{3}\right) \cdot \prod_{v \in S}^{}  Z_{v}^{*}\left(0, W_{v}^{\psi},
\Phi_{s, v}\right) \\
= \int_{Z_{G}\left(\mathbb{A}\right) G\left(k\right) \backslash
G\left(\mathbb{A}\right)}^{} E^{*}\left(g, 0, \Psi_{s}\right) F\left(g\right) \,
\mathrm{d}g,\end{multline}
%
where 
%
\[Z_{v}^{*}\left(s, W_{v}^{\psi}, \Phi_{s, v}\right) = \begin{cases}
\tilde{Z}_{v}\left(s, W_{v}^{\psi}, \Phi_{s, v}\right) & \text{if } v \in
S_{f},\\ Z_{v}\left(s, W_{v}^{\psi}, \Phi_{s, v}\right) & \text{if } v \in
S_{\infty}. \end{cases}\]
%
We make the abbreviation 
%
\[Z^{*}\left(F, \Phi\right) = \prod_{v \in S}^{} Z_{v}^{*}\left(0, W_{v}^{\psi},
\Phi_{s, v}\right).\]
%
\end{itemize}

\section{The Weil representation for similitudes.}

\begin{itemize}
    
\item Let $B$ be a quaternion algebra over $k$, and let $V = B$ be a $4$
dimensinal quadratic space over $k$ where the quadratic form is given by
$Q\left(x\right) = \alpha v\left(x\right)$, where $v$ is the reduced norm on $B$
and $\alpha \in k^{\times}$

\item Let $H = \text{GO}\left(V\right)$ and $H_{1} = \text{O}\left(V\right)$.
Let $G = \text{GSp}_{6}$ and let $G_{1} = \text{Sp}_{6}$.

\item Let 
%
\[R = \left\{\left(h, g\right) \in H \times G \; | \; v\left(h\right) =
v\left(g\right)\right\}.\]
%
There exists an extension of the standard Weil representation $\omega =
\omega_{\psi}$ of $H_{1}\left(\mathbb{A}\right) \times
G_{1}\left(\mathbb{A}\right)$ on the Schwartz space
$S\left(V\left(\mathbb{A}\right)^{3}\right)$ to $R\left(\mathbb{A}\right)$.

\item Let 
%
\[G\left(\mathbb{A}\right)^{+} := \left\{g \in G\left(\mathbb{A}\right) \; | \;
v\left(g\right) \in v\left(H\left(\mathbb{A}\right)\right)\right\}\]
% 
\[= \left\{g \in G\left(\mathbb{A}\right) \; | \; v\left(g\right)_{v} > 0, \;
\forall v \in \Sigma_{\infty}\left(V\right)\right\}.\]
%
\item For $g \in G\left(\mathbb{A}\right)^{+}$, and $\varphi \in
S\left(V\left(\mathbb{A}\right)^{3}\right)$, and for $V$ anisotropic over $k$,
the theta integral is defined by 
%
\[I\left(g, \varphi\right) = \int_{H_{1}\left(k\right) \backslash
H_{1}\left(\mathbb{A}\right)}^{} \theta\left(h_1h, g; \varphi\right) \,
\mathrm{d}h_{1},\]
%
where $h \in H\left(\mathbb{A}\right)$ with $v\left(h\right) = v\left(g\right)$.
An alternate definition has to made if $V$ is not anisotropic over $k$.
$I\left(g, \varphi\right)$ has a unique extension to a left
$G\left(k\right)$-invariant function on $G\left(\mathbb{A}\right)$.

\end{itemize}

\section{The Siegel-Weil formula for $\left(\text{GO}\left(V\right),
\text{GSp}_{6}\right)$.}

\section{Proof of Jacquet's conjecture.}

\begin{itemize}
    
\item Applying the Siegel-Weil formula for similitudes to the basic identity
(1), we obtain 
%
\[L\left(\frac{1}{2}, \pi_1 \otimes \pi_2 \otimes \pi_3\right) \cdot
Z^{*}\left(F, \Phi\right)\]
%
\[= \int_{Z_{G}\left(\mathbb{A}\right)G\left(k\right) \backslash
G\left(\mathbb{A}\right)}^{} E^{*}\left(g, 0, \Psi_{s}\right) F\left(g\right) \,
\mathrm{d}g\] 
%
\[= 2 \zeta_{k}\left(2\right)^{2} \sum_{V}^{}
\int_{Z_{G}\left(\mathbb{A}\right)G\left(k\right) \backslash
G\left(\mathbb{A}\right)} I\left(g, \varphi^{V}\right) F\left(g\right)
\mathrm{d}g.\]

\item Next, we want to apply the seesaw identity. We set 
%
\[H = \text{GO}\left(V\right)\]
%
%
\[\textbf{H} = \left\{\left(h_1, h_2, h_3\right) \in H^{3} | v\left(h_{1}\right)
= v\left(h_2\right) = v\left(h_3\right)\right\}\]
%
For $F$ a cuspidal automorphic form on $\textbf{G}\left(\mathbb{A}\right)$ and
for $h \in \textbf{H}\left(\mathbb{A}\right)$, let 
%
\[I\left(h, \varphi; F\right) = \int_{\textbf{G}_{1}\left(k\right) \backslash
\textbf{G}_1\left(\mathbb{A}\right)}^{} \theta\left(h, \textbf{g}_1\textbf{g};
\varphi\right) F\left(\textbf{g}_{1} \textbf{g}\right) \,
\mathrm{d}\textbf{g}_{1}.\]
%
The seesaw identity is
%
\[\int_{Z_{G}\left(\mathbb{A}\right) \textbf{G}\left(k\right) \backslash
\textbf{G}\left(\mathbb{A}\right)}^{} I\left(g, \varphi\right) F\left(g\right)
\, \mathrm{d}g = \int_{Z_{H}\left(\mathbb{A}\right)H\left(k\right) \backslash
H\left(\mathbb{A}\right)}^{} I\left(h, \varphi ; F \right) \, \mathrm{d}h.\]
%
\item 
\end{itemize}


\end{document}
