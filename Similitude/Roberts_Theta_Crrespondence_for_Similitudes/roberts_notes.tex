\documentclass[12pt]{article}

\usepackage{microtype}

\usepackage{amsmath}
\usepackage{amsfonts}
\usepackage{MnSymbol}

\usepackage{tikz-cd}

\title{Notes on ``The Theta Correspondence for Similitudes'' By Brooks Roberts}
\author{Nick Pilotti}
\date{\today}

\begin{document}

\maketitle

\textbf{Fixing notation.}
\begin{itemize}
\item $k$ a nonarchimedean local field.
\item X a finite dimensional nondegenerate symmetric bilinear space over $k$.
\item Y a finite dimensional nondegenerate symplectice bilinear space over $k$.
\item $p$ the projection $\text{Mp}\left(X \otimes_{k} Y\right) \rightarrow
\text{Sp}\left(X \times_{k} Y\right)$.
\item $r$ the smooth Weil representation of $\text{Mp}\left(X \times_{k}
Y\right)$ (corresponding to some fixed nontrivial additive character of $k$).
\end{itemize}

\textbf{The theta correspondence.}
 
 The restriction of $r$ to $p^{-1}\left(\text{O}\left(X\right)\right)p^{-1}\left(\text{Sp}\left(Y\right)\right)$
definites a correspondence between the smooth admissible duals of
$p^{1}\left(\text{O}\left(X\right)\right)$ and
$p^{-1}\left(\text{Sp}\left(Y\right)\right)$. When the residue characteristic of
$k$ is odd, this correspondence satisfies strong Howe duality.

\textbf{Biblographic notes for ease of reference.} 

The Weil representation was first constructed in Weil (1964) ``Sur certain
groupes d'opérateurs unitaires.'' The theta correspondence was introduced in
Howe (1979) ``$\theta$-series and invariant theory.'' Howe duality was proved
for archimedean local fields in Howe (1989) ``Transcending classical invariant
theory'' and for $p$-adic fields with $p$ odd in Waldspurger (1990)
''Démonstration d'une conjecture de dualité de Howe danse le case p-adique, $p
\neq 2$.'' A classic reference (in French) is MVW (1987) ''Correspondance de
Howe sur un corps p-adique.'' A useful set of notes is Kulda (1996) ''Notes on
the Local Theta Correspondence'' and Gan's AWS notes. A very comprehensive
modern textbook is GKT ''The Local Theta Correspondence.''

\textbf{Summary of the paper.}

The two main approaches to a theta correspondence for
similitudes are ''essentially the same'' and a version of strong Howe duality
holds for both constructions. The two main constructions are:

\begin{itemize}
\item Extend the restriction of $r$ to
$p^{1}\left(\text{O}\left(X\right)\right)p^{-1}\left(\text{Sp}\left(Y\right)\right)$
to a representation $\omega$ of a larger group involving similitudes.
\item Induce the restriction of $r$ to
$p^{-1}\left(\text{Sp}\left(Y\right)\right)$ to obtain a representation $\Omega$
that involves similitues.
\end{itemize}

\textbf{More notation.}

\begin{itemize}

\item $H = \text{GO}\left(X\right)$.

\item If $\text{dim}_{k} X$ is even, let $G' = \text{GSp}\left(Y\right)$.

\item If $\text{dim}_{k} X$ is odd, let $G'$ be a certain two-fold cover of
$\text{GSp}\left(Y\right)$.

\item For $g \in G'$, let $\lambda\left(g\right)$ be the similitude factor of
the projection of $g$ to $\text{GSp}\left(Y\right)$. Let $\lambda\left(h\right)$
be the similitude factor of $h \in H$.

\item Let $G$ be the subgroup of $g \in G'$ such that $\lambda\left(g\right) \in
\lambda\left(H\right)$ (recall that the map $\lambda$ is surjective on
$\text{GSp}\left(Y\right)$ but not necessarily on $\text{GO}\left(X\right)$).

\item Let $G_1$ and $H_1$ be the subgroups of $g \in G$ and $h \in H$ such that
$\lambda\left(g\right) = 1$ and $\lambda\left(h\right) = 1$, respectively (so
$H_{1}$ is just $\text{O}\left(X\right)$ and $G_{1}$ is
$\text{Sp}\left(Y\right)$ in case $\text{dim}_{k} X$ is even.

\item $\omega$ is a representation of the group 

\[R = \left\{\left(g, h\right) \in G' \times H : \lambda\left(g\right) =
\lambda\left(h\right)\right\}\]

and $\Omega$ is a representation of $G' \times H$.

\end{itemize}

\textbf{Detailed summary.}

\begin{itemize}
    
\item Section $1$ defines Howe duality, multiplication preservation, and strong
Howe duality. It also shows that Howe duality and multiplication preservation
are equivalent to strong Howe duality.

\item Sections $2$ and $3$ construct and relate $\omega$ and $\Omega$. $\Omega$
is obtained from $\omega$ via compact induction: 

\[\Omega \cong \text{c-Ind}_{R}^{G' \times H} \omega.\]

\item Section 4 studies the correspondence defined by $\omega$. In particular,
it proves that the analogues of Howe duality and multiplicity preservation hold.

\item Sections $5$ and $6$ consider the consequences of section $4$ for
$\Omega$. It gives a condition for Howe duality called theta dichotomy. It shows
that Howe duality does not hold for $\Omega$ in the stable range. When
$\text{dim}_{k} X \leq \text{dim}_{k} Y$, strong Howe duality for $\Omega$ is
expected to hold.

\end{itemize}

\textbf{Some more notation.}


\begin{itemize}
    
\item Let $J$ be a group of td-type. This means that $J$ is a topological group
and every neighborhood of the identity element of $J$ contains a compact open
subgroup. For such groups, Schur's lemma holds.

\item Let $\text{Irr}\left(J\right)$ be the set of equivalence classes of smooth
admissible irreducible representations of $J$.

\item If $\pi \in \text{Irr}\left(J\right)$ then $\pi^{\vee} \in
\text{Irr}\left(J\right)$ denotes its contragredient representation.

\item A character of $J$ is a continuous homomorphism from $J$ to
$\mathbb{C}^{\times}$.

\item If $L$ is a closed normal subgroup of $J$, $\pi \in
\text{Irr}\left(L\right)$ and $g \in J$, then $g \pi \in
\text{Irr}\left(L\right)$ is the representation with the same space as $\pi$ and
action defined by $\left(g \pi\right)\left(h\right) = \pi\left(g^{-1}hg\right)$,
and $J_{\pi}$ is the subgroup of $g \in J$ such that $g \pi \cong \pi$.

\item Let $\left(\;,\;\right)_{k}$ denote the Hilbert symbol of $k$.

\end{itemize}

\section{Howe duality and multiplicity preservation}

\textbf{Existence and uniquenss of the big theta lift.}

Let $A$ and $B$ be groups of td-type, with countable bases. Let $\left(\rho,
\mathcal{U}\right)$ be a smooth representation of $A \times B$. Let $\pi \in
\text{Irr}\left(A\right)$. Define 

\[\mathcal{U}\left(\pi\right) = \mathcal{U} / \bigcap_{t \in
\text{Hom}_{A}\left(p, \pi\right)} \text{ker}\left(t\right).\]

Via $\rho$, $A \times B$ acts on $\mathcal{U}\left(\pi\right)$. Call this
representation $\rho\left(\pi\right)$. By [MVW] there exists a smooth
representation $\Theta\left(\pi\right)$ of $B$, unique up to isomorphism, such
that 

\[\rho\left(\pi\right) \cong \pi \otimes_{\mathbb{C}} \Theta\left(\pi\right)\]

as representations of $A \times B$. Analogous remarks apply for elements of
$\text{Irr}\left(B\right)$.

\textbf{Strong Howe duality.}

Let $\mathcal{R}\left(A\right)$ be the set of equivalence classes of $\pi \in
\text{Irr}\left(A\right)$ such that $\mathcal{U}\left(\pi\right) \neq 0$ and
define $\mathcal{R}\left(B\right)$ similarly. We say that strong Howe duality
holds for $\rho$ if for every $\pi \in \mathcal{R}\left(A\right)$ the
representation $\Theta\left(\pi\right)$ has a unique nonzero irreducible
quotient $\theta\left(\pi\right) \in \mathcal{R}\left(B\right)$, and for every
$\tau \in \mathcal{R}\left(B\right)$ the representation
$\Theta\left(\tau\right)$ has a unique nonzero irreducible quotient
$\theta\left(\tau\right) \in \mathcal{R}\left(A\right)$.

\textbf{Howe duality.} We say that Howe duality holds for $\rho$ if the set 

\[\mathcal{R}\left(A \times B\right) = \left\{\left(\pi, \tau\right) \in
\mathcal{R}\left(A\right) \times \mathcal{R}\left(B\right) : \text{Hom}_{A
\times B}\left(\rho, \pi \otimes_{\mathbb{C}} \tau\right) \neq 0\right\}\]

is the graph of a bijection between $\mathcal{R}\left(A\right)$ and
$\mathcal{R}\left(B\right)$. Equivalently, Howe duality holds for $\rho$ if and
only if $(1)$ every $\pi \in \mathcal{R}\left(A\right)$ occurs as the first
entry of an element of $\mathcal{R}\left(A \times B\right)$ and every $\tau \in
\mathcal{R}\left(B\right)$ occurs as the second entry of an element of
$\mathcal{R}\left(A \times B\right)$; and (2) for all $\pi \in
\text{Irr}\left(A\right)$ and $\tau_1, \tau_2 \in \text{Irr}\left(B\right)$, 

\[\text{Hom}_{A \times B}\left(\rho, \pi \otimes_{\mathbb{C}} \tau_1\right) \neq
0, \; \text{Hom}_{A \times B}\left(\rho, \pi \otimes_{\mathbb{C}} \tau_2\right)
\neq 0 \implies \tau_1 \equiv \tau_2;\]

for all $\pi_1, \pi_2 \in \text{Irr}\left(A\right)$ and $\tau \in
\text{Irr}\left(B\right)$, 

\[\text{Hom}_{A \times B}\left(\rho, \pi_1 \otimes_{\mathbb{C}} \tau\right) \neq
0, \; \text{Hom}_{A \times B}\left(\rho, \pi_{2} \otimes_{\mathbb{C}}
\tau\right) \neq 0 \implies \pi_{1} \cong \pi_{2}.\]

\textbf{Multiplicity preservation.} We show that multiplicity preservation holds
for $\rho$ if for all $\pi \in \text{Irr}\left(A\right)$ and $\tau \in
\text{Irr}\left(B\right)$, 

\[\text{dim}_{\mathbb{C}} \text{Hom}_{A \times B}\left(\rho, \pi \otimes_{C}
\tau\right) \leq 1.\]

\textbf{Proposition 1.1.} 

Strong Howe duality holds for $\rho$ if and only if
Howe duality and multiplicity preservation hold for $\rho$. If strong Howe
duality holds for $\rho$, then the map $\theta : \mathcal{R}\left(A\right)
\rightarrow \mathcal{R}\left(B\right)$ is the bijection given by Howe duality.

\textbf{Proposition 1.2. (used in section 6)}

Assume that $A$ is contained in a group $A'$ of td-type with countable basis as
a closed normal subgroup of index two. Let $a$ be a representative for the
nontrivial coset of $A'/A$. Let $\rho' = \text{Ind}_{A \times B}^{A' \times B}
\rho$. All of the above definitions apply with $\rho'$ in place of $\rho$. If
Howe duality holds for $\rho$, then Howe duality holds for $\rho'$ if and only
if $\mathcal{R}\left(A\right) \cap a \cdot \mathcal{R}\left(A\right) =
\emptyset$. If strong Howe duality holds for $\rho$, then strong Howe duality
holds for $\rho'$ if and only if $\mathcal{R}\left(A\right) \cap a \cdot
\mathcal{R}\left(A\right)$.

\section{The groups}

\textbf{Polarization of symplectic space.} Let $\left\llangle \;,\;
\right\rrangle$ be a finite dimensional nondegenerate symplectic vector space
over $k$. Assume $\mathbb{W} \neq 0$. There exists a basis $\mathbb{W}$ relative
to which the symplectic form has a the matrix representation 

\[\begin{pmatrix} 0 & I_{n} \\ I_{n} & 0 \end{pmatrix}\]

where $I_{n}$ is the $n \times n$ identity matrix and $2n =
\text{dim}\left(\mathbb{W}\right)$. Take $\mathbb{U}$ to be the subspace
generated by the first $n$ vectors in this basis (the symplectic form restricted
to $\mathbb{U}$ is $0$). We see that $\mathbb{U}^{*}$ can be identified with the
subspace generated by the remaining $n$ vectors. Hence, $\mathbb{W} \cong
\mathbb{U} \oplus \mathbb{U}^{*}$. We write elements of
$\text{GL}\left(\mathbb{W}\right)$ with repsect to this basis.

\textbf{Definition and structure of $\text{GSp}\left(\mathbb{W}\right)$.} Let
$\text{GSp}\left(\mathbb{W}\right)$ be the subgroup of
$\text{GL}\left(\mathbb{W}\right)$ such that there exists $\lambda \in
k^{\times}$ such that $\left\llangle gw, gw' \right\rrangle = \lambda
\left\llangle w, w'\right\rrangle$ for all $w, w' \in \mathbb{W}$. If $g \in
\text{GSp}\left(\mathbb{W}\right)$, then such a $\lambda$ is unique, and it will
be denoted $\lambda\left(g\right)$. Let $\text{Sp}\left(\mathbb{W}\right)$ be
the subgroup of $g \in \text{GSp}\left(\mathbb{W}\right)$ such that
$\lambda\left(g\right) = 1$. $\lambda$ is a homomorphism and the map 

\[y \mapsto \begin{pmatrix} 1 & 0 \\ 0 & y \end{pmatrix}\]

is a splitting for the short exact sequence 

\[1 \rightarrow \text{Sp}\left(\mathbb{W}\right) \rightarrow
\text{GSp}\left(\mathbb{W}\right) \rightarrow k^{\times} \rightarrow 1.\]

Hence, $\text{GSp}\left(\mathbb{W}\right) \cong k^{\times} \ltimes
\text{Sp}\left(\mathbb{W}\right)$ and $k^{\times}$ acts on
$\text{Sp}\left(\mathbb{W}\right)$ by conjugation.

\textbf{The metaplectic similitude group.} 

We denote by $\text{GMp}\left(\mathbb{W}\right)$ the metaplectic cover of
$\text{GSp}\left(\mathbb{W}\right)$, constructed in more detail in the text.
There is also a two-fold cover of $\text{GSp}\left(\mathbb{W}\right)$ denoted
$\widehat{\text{GSp}\left(\mathbb{W}\right)}$ and an inclusion of
$\widehat{\text{GSp}\left(\mathbb{W}\right)}$ in
$\text{GMp}\left(\mathbb{W}\right)$ such that the following diagram commutes: 

\[
\begin{tikzcd} 
\text{GMp}\left(\mathbb{W}\right) \arrow{r} & \text{GSp}\left(\mathbb{W}\right) \\
\widehat{\text{GSp}\left(\mathbb{W}\right)} \arrow{u} \arrow{r} &
\text{GSp}\left(\mathbb{W}\right) \arrow{u}.
\end{tikzcd}
\]

\textbf{$\text{GO}\left(X\right)$ and $\text{GSp}\left(Y\right)$.}

Let $\left(X, \left(\;, \;\right)\right)$ be a nondegenearte symmetric bilinear
space over $k$ of dimension $m$, and let $\left(Y, \left\langle \;, \;
\right\rangle\right)$ be a nondegenerate symplectic bilinear space over $k$ of
dimension $2n$. For the remainder we will assume that 

\[\left(\mathbb{W}, \left\llangle \;, \; \right\rrangle\right) = \left(X,
\left(\;, \;\right)\right) \otimes_{k} \left(Y, \left\langle \;,\;
\right\rangle\right),\]

and there is a complete polarization $Y = U \oplus U^{*}$ such that $\mathbb{U}
= X \otimes_{k} U$ and $\mathbb{U}^{*} = X \otimes_{k} U^{*}$. Let
$\text{GO}\left(X\right)$ be the similitude group of $X$ defined in a similar
way as for symplectic spaces. There are inclusions 

\[\text{GO}\left(X\right) \rightarrow \text{GSp}\left(\mathbb{W}\right), \;
\text{GSp}\left(Y\right) \rightarrow \text{GSp}\left(\mathbb{W}\right).\]

The elements of $p^{-1}\left(\text{GO}\left(X\right)\right)$ and
$p^{-1}\left(\text{GSp}\left(Y\right)\right)$ in general do not commute. We have
the following results on the structure of these groups.

\textbf{Proposition 2.1.}

If $m$ is even, then $p^{-1}\left(\text{GSp}\left(Y\right)\right)$ is trivial as
an extension of $\text{GSp}\left(Y\right)$ by $\mathbb{C}^{1}$. If $m$ is odd,
then $p^{-1}\left(\text{GSp}\left(Y\right)\right)$ is the metaplectic cover of
$\text{GSp}\left(Y\right)$.

\textbf{Proposition 2.3.} Assume that the residual characteristic of $k$ is odd.
If $m$ is odd, $m = 2$, or $m = 4$ and $X$ is anisotropic, then
$p^{-1}\left(\text{GO}\left(X\right)\right)$ is trivial as an extension of
$\text{GO}\left(X\right)$ by $\mathbb{C}^{1}$. If $m$ is even, $m \geq 4$ and
$X$ is not four dimensional and anisotropic, then
$p^{-1}\left(\text{GO}\left(X\right)\right)$ is trivial if and only if the
character of $\text{GSO}\left(X\right)$ defined by $h \mapsto \left(-1,
\lambda\left(h\right)\right)_{k}^{n}$ is trivial.

\section{The representations}

\textbf{Proposition 3.1.} Consider the subgroup of
$\text{GSp}\left(\mathbb{W}\right)$ generated by the elements 

\[\begin{pmatrix} \lambda\left(h\right)^{-1} & 0 \\ 0 & 1 \end{pmatrix} \left(h
\otimes 1\right)\] for $h \in \text{GO}\left(X\right)$. This subgroup is
isomorphic to $\text{GO}\left(X\right)$. Its preimage under $p$ is trivial as an
extension of $\text{GO}\left(X\right)$ by $\mathbb{C}^{1}$.

\textbf{Lemma 3.2. (Fundamental identity).} 

Let $L$ be a splitting of the preimage of the subgroup from Proposition $3.1$.
If $h \in \text{GO}\left(X\right)$ and $g \in
p^{-1}\left(\text{Sp}\left(Y\right)\right)$ then 

\[L\left(h\right)gL\left(h\right)^{-1} = g^{\lambda\left(h\right)^{-1}}.\]

\textbf{Proposition 3.3. (Shimizu-Harris-Kudla).} 

Define an action of $H$ on $G_1$ by $h \cdot g = g^{\lambda\left(h\right)^{-1}}$
and form the semidirect product $G_1 \rtimes H$. The map $G_{1} \rtimes H
\rightarrow \text{Mp}\left(\mathbb{W}\right)$ defined by $\left(g, h\right)
\mapsto g L\left(h\right)$ is a homomorphism. Let 

\[R = \left\{\left(g, h\right) \in G' \times H : \lambda\left(g\right) =
\lambda\left(h\right)\right\}.\]

The map $R \rightarrow G_{1} \rtimes H$ defined by 

\[\left(g, h\right) \mapsto \left(gd\left(\lambda\left(g\right)\right)^{-1}, h\right)\]

is a homomorphism. Thus, the composition $\omega$ 

\[R \rightarrow G_{1} \rtimes H \rightarrow \text{Mp}\left(\mathbb{W}\right)
\rightarrow \text{Aut}_{\mathbb{C}}\left(\mathcal{S}\right)\]

is a homomorphism. This representation is smooth.

$\omega$ is called the \textbf{extended Weil representation associated to $X$ and
$Y$}. We have 

\[\omega\left(g, h\right) = r\left(gh\right).\]

\textbf{Proposition 3.4.} Let $\Omega = \text{c-Ind}_{G_{1}}^{G'} r$. For each
$h \in H$, define an operator $\Omega\left(h\right)$ on the space $\mathcal{T}$
of $\Omega$ by 

\[\left(\Omega\left(h\right)f\right)\left(g\right) =
r\left(L\left(h\right)\right) \cdot
f\left(d\left(\lambda\left(h\right)\right)^{-1}g\right).\]

Then the map 

\[\Omega : G' \times H \rightarrow \text{Aut}_{\mathbb{C}}\left(\mathcal{T}\right)\]

defined by $\left(g, h\right) \mapsto \Omega\left(g\right)\Omega\left(h\right)$
is a homomorphism. This representation is smooth. We call $\Omega$ the
\textbf{induced Weil representation}.

\textbf{Proposition 3.5.} We have 

\[\Omega \cong \text{c-Ind}_{R}^{G' \times H} \omega.\]

\textbf{The group $G$ and the representation $\Omega^{+}$.} Let $G$ be the group
of $g \in G'$ such that $\lambda\left(g\right) \in \lambda\left(H\right)$. A
smooth representation $\left(\Omega^{+}, \mathcal{T}^{+}\right)$ of
$\Omega|_{G}$ is defined, and turns out to be isomorphic to $\text{c-Ind}_{R}^{G
\times H} \omega$.

\end{document}
